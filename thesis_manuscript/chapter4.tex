\chapter{Grid Search}
\label{ch4}
This chapter walks through the methodology used to select the best model and parameters to forecast 4 hours of $C_{n}^{2}$ given prior environmental measurements, then the statistical analysis performed as a justification for the model selection.

\section{Methodology}
Given a problem to model with machine learning there are model hyperparameters to adjust in infinitely many combinations, each of which can impact model performance. Just a few examples of these hyperparameters are the optimization algorithm, the learning rate of the optimization algorithm, the number of layers in a model architecture and the number of nodes per layer. Given the many combinations of a model's hyperparameters, a careful method to determine the best combination must be employed. The definition of the ``best" model is the model which results in the best performance when applied to the validation dataset, a subset of the entire dataset specifically held from training for hyperparameter tuning. By holding back the validation dataset, an unbiased optimization of the hyperparemeters can be performed, then those parameters are used in the model applied to the test dataset for an unbiased evaluation of the final model.

There are many methods of hyperparameter optimization, but common methods are grid search and random search. Grid search, or a parameter sweep, is an exhaustive search through manually specified hyperparameters. If iterating over only one hyperparameter, for example three different learning rates, a model is trained with the first learning rate and it's performance on the validation dataset is recorded, then the model is trained again in the same fashion but with the second learning rate and the performance on the validation dataset is recorded, and finally this is done again for the third learning rate. After, the performance of the models with the three learning rates are compared and the best learning rate is the learning rate used by the model that performed best. This method can quickly explode in computation time as the number of hyperparameters to iterate increases. For example, iterating over two hyperparameters of three values each results in nine combinations of hyperparameters. The number of combinations is defined as the multiple of the number of values across each hyperparameter, so if there are five hyperparameters with 1, 2, 3, 4, and 5 values, then the total number of combinations is $1 \times 2 \times 3 \times 4 \times 5 = 120$ combinations. The random search replaces the exhaustive grid search by randomly selecting hyperparameters within defined bounds. An algorithm randomly selects the hyperparameters and models are trained with the different combinations, then applied to the validation dataset to evaluate performance, and the hyperparameters associated with best performing model are the best hyperparameters. A benefit of the random search is the selection of parameter combinations that might not be defined in a grid search.

In this work the grid search is employed because prior knowledge of hyperparameters is unknown. Thus, the exhaustive and highly inclusive grid search is necessary to explore a wide range of combinations. The grid search iterates over five parameters. The outermost parameter is the four fundamental architectures: MLP, simple RNN, GRU, and LSTM. The MLP is a common machine learning architecture and serves as a baseline. The simple RNN, GRU, and LSTM are variants of the general RNN and are searched to find if a specific variant is better or worse than the others. The next parameter is the input sequence variables used by the model. From Section \ref{sec:wx_seq_hist}, the available input sequence features (variables) are prior temperature, pressure, relative humidity, wind speed, solar irradiance, and $C_{n}^{2}$ measurements. The ``input sequence features" parameter iterates over which features (variables) to train the model. Since there are six available features there are $2^6 = 64$ total combinations. However, a threshold is set to train on a minimum of four features which brings the total number of input feature combinations to 22. This threshold is set to discourage the model from memorizing one or two features, and for a significant reduction in computation time. The third search parameter is the input sequence length. Independent of the features (variables), the amount of information available to the model is dependent on the time-length of the input sequences. Whether the model performs best with only 4 hours of input data or 16 hours of input data is highly relevant knowledge. Thus, four values of the input sequence lengths are searched: 4, 8, 12, and 16 hours. These lengths are chosen to be $1\times$, $2\times$, $3\times$, and $4\times$ the 4 hour forecast length. Note that the train, validation, and test datasets are carefully formatted so the same forecasts are trained, validated, and tested regardless of the input sequence length to avoid an instance where a 4 hour sequence might exist for a forecast but a 16 hour sequence is not available. The fourth parameter searched is the number of hidden layers in each architecture: 1 or 2. The fifth and final searched parameter is the number of hidden nodes per hidden layer: 10 through 50 in steps of 10. The final two parameters essentially search over the number of parameters in the model with some variation in the interaction of those parameters. Each fundamental architecture in total iterates over $22 \times 4 \times 2 \times 5 = 880$ combinations of parameters. Due to the stochastic nature of model training, a single model could perform significantly different than another model trained with the same parameters, thus a total of 10 models are trained per combination per fundamental architecture to ensure the stability of results. In this search a total of $880 \times 4 \times 10 = 35,200$ models are trained.

For each model trained in the grid search the following parameters are fixed: mini-batch size, optimization algorithm, initial learning rate, learning rate decay (step and decay factor), and weight decay. The mini-batch size, the number of training examples used per model parameter update, is set to 32 yielding a total of 30 parameter updates (933/32) per epoch (iteration through the entire dataset). The optimization algorithm is AdamW, one of the most popular optimization algorithms used today (\textcolor{blue}{Go in depth about the algorithm here, or reference the background section? Reference paper on Adam, paper on correct implementation of regularization term, AdamW, and pytorch implementation}). The initial learning rate is 0.01, and decays by a factor of 10 every 10 epochs. This results in learning rates 0.01, 0.001, 1e-4, 1e-5, and 1e-6 from epochs 1 - 10, 11 - 20, 21 - 30, 31 - 40, and 40 - 50, respectively. The high initial learning rate is to ensure suitably-high gradients are back-propagated through the model architecture to allow each model 10 epochs (300 total parameter updates) to escape any local minima. This training method consistently leads to a strong model convergence in only a few seconds. The weight decay is a regularization technique applied to the optimization algorithm and is 0.001 \textcolor{blue}{there is no reason why I chose this specific value, 1e-3; I just wanted to employ a bit of regularization to avoid overfitting}.

\section{Results}
The analyses of the grid search are performed independently on each architecture. From the four grid searches, four 2d-arrays of RMSE loss scores, the performance metric between validation truth and output $log_{10}(C_{n}^{2})$, are available for analysis. The 2d-arrays are of shape $880 \times 10$: 880 parameter combinations and 10 models each. From these four 2d arrays, the average and standard deviation (1 degree of freedom) of the RMSE scores are calculated per parameter combination yielding four arrays of 880 averages and standard deviations of RMSE scores. The standard error, or the standard deviation of the mean, is calculated by dividing the standard deviation by the square root of the number of samples, 10 in this case \textcolor{blue}{add equations here!}. From these statistics the best model is determined and significance quantified.

\textcolor{blue}{Describe the process of finding the best model: compare the averages}.
\textcolor{blue}{Describe the plots in Figure \ref{fig:grid_search_results} which sort the averages per architecture, then the average with the standard error}.
\begin{figure}[h!]
	\centering
	\subfloat[Full set of iterations\label{fig:grid_search_results_a}]{
		\includegraphics[width=0.49\textwidth]{average_model_performance_wide.png}
	}
	\subfloat[Top 10 iterations\label{fig:grid_search_results_b}]{
		\includegraphics[width=0.49\textwidth]{average_model_performance_narrow.png}
	}
	\hfill
	\caption{Grid search results.}
	\label{fig:grid_search_results}
\end{figure}
\textcolor{blue}{Describe student's t-test and the plot in Figure \ref{fig:students_t-test}}.
\begin{figure}[h!]
	\centering
	\includegraphics[width=0.7\textwidth]
	{students_t-test.png}
	\hfill
	\caption{Grid search results.}
	\label{fig:students_t-test}
\end{figure}
\textcolor{blue}{Describe Figure \ref{fig:variable_sets_analysis} and how it is a variable sensitivity analysis.}
\begin{figure}[h!]
	\centering
	\subfloat[Best 10\%\label{fig:variable_sets_analysis_a}]{
		\includegraphics[width=0.49\textwidth]{bar_variable_sets_best.png}
	}
	\subfloat[Worst 10\%\label{fig:variable_sets_analysis_b}]{
		\includegraphics[width=0.49\textwidth]{bar_variable_sets_worst.png}
	}
	\hfill
	\caption{Best 10\% and worst 10\% variable sets.}
	\label{fig:variable_sets_analysis}
\end{figure}
\textcolor{blue}{Drite reasons for picking the best GRU model: best average score, standard error is similar to next best MLP model, shorter data requirement, no real reason to pick MLP}.