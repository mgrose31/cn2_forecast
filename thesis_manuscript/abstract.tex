Optical (atmospheric) turbulence ($C_{n}^{2}$) is a highly stochastic process that can apply many adverse effects on imaging and laser propagation systems. Modeling atmospheric turbulence conditions has been proposed by physics-based models but they are unable to capture the many cases. Recently, machine learning surrogate models have been used to learn the relationship between local environmental (weather) and turbulence conditions. These models predict a turbulence strength at time $t$ from weather at time $t$. This thesis proposes a technique to forecast four hours of future turbulence conditions at 30-minute intervals from prior environmental parameters using artificial neural networks. First, local weather and turbulence measurements are formatted to pairs of input sequence and output forecast. Next, a grid search is performed to find the best combination of model architecture and training parameters. The architectures investigated are the Multilayer Perceptron (MLP) and three variants of the Recurrent Neural Network (RNN). Finally, the selected model is applied to the test dataset and analyzed. It is shown that the model has generally learned the relationship between prior environmental and future turbulence conditions.